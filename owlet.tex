\documentclass[11pt]{article}
\usepackage[margin=1in, paperwidth=8.5in, paperheight=11in]{geometry}
\usepackage{graphicx}
\setlength\intextsep{0pt}
\usepackage{enumitem}

\setitemize[1]{itemsep=-5pt,topsep=0pt}

\pagestyle{empty}
\setlength{\belowcaptionskip}{-10pt}

\usepackage{graphicx}
\usepackage{hyperref}
\usepackage[
type={CC},
modifier={by},
version={4.0},
]{doclicense}


\pagestyle{empty}
\begin{document}
\begin{centering}
	
\textbf{\LARGE Parliament of Owlets Hat}
\medskip

{\large Veronica Boyce}

\smallskip

\end{centering}


A double knit hat in fingering yarn with 6 owls on the side and a flower shape on the top. Owls use cables, nupp eyes, and color changes. Hat uses increases and decreases for shaping. 
\smallskip

\textbf{Caveats:} This pattern assumes a working familiarity with double knitting and willingness to use hard techniques. This pattern uses the following techniques, which are not explained.
\begin{itemize}
	\item 4 pair x 5 pair cables
	\item nupps
	\item lifted increases
	\item single decreases
\end{itemize}

\smallskip
\textbf{Yarn:} ZZ grams of each of two colors of fingering weight yarn (approx YY yards). Shown in Tenderfoot Superfine (75\% merino lambswool, 25\% polyamide) in colors Linen and Deep Russet. 

\smallskip
\textbf{Needles:} \begin{itemize}
	\item size 4 needles (24 inch circular)
	\item cable needle (J shape recommended)
	\item tapestry needle
	\item size 2 needle (dpn, tip from interchangeable or circular) for nupps
\end{itemize}

\smallskip
\textbf{Sizing:} As shown, the pattern is repeated 6 times, for a finished diameter of AA and height of BB, to fit a QQ diameter head. To resize, you could work fewer/more repeats of the pattern, and/or adjust the spacing between owls. If you adjust spacing between owls, you will need to modify the increases in chart B and the decreases at the beginning of chart D. I recommend doing chart A over 2/3 the number of stitches used in Chart C. 

\smallskip
\textbf{Swatching:} I recommend working one copy of chart C (plus stockinette border) flat as coaster to get familiar with the owl technique and check sizing. 

\smallskip
\textbf{Notes:}
\begin{itemize}
	\item \textbf{Nupps:} When working the owl eyes, it's important to get the nupp loops loose enough. I recommend holding a small needle (I used size 2 interchangeable tip) double with the right hand needle. Work the nupp loops over both needles, work other stitches in that row only on the normal size 4 needle. 
	\item \textbf{Cables:} This pattern has 4 pair x 5 pair cables (which feel like 8 x 10 cables!) Slipping stitches purlwise and use a J shaped cable needle to get stitches reordered and back on the left hand needle, then work the cable. It will feel very tight. 
\end{itemize}

\smallskip

\textbf{Pattern:}
Cast on 96 pairs of AB color. 

Work Chart A six times around for 15 rows (96 pairs). 

Work Chart B six times around (144 pairs). 

Work Chart C six times around (144 pairs). 

Work Chart D six times around (6 pairs). 

Cinch stitches with yarn ends, knot, and secure ends. Block if desired. 


\doclicenseThis
Contact: vero.boyce@gmail.com
\end{document}